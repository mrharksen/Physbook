%% Copyright 2009 Ivan Griffin
%
% This work may be distributed and/or modified under the
% conditions of the LaTeX Project Public License, either version 1.3
% of this license or (at your option) any later version.
% The latest version of this license is in
%   http://www.latex-project.org/lppl.txt
% and version 1.3 or later is part of all distributions of LaTeX
% version 2005/12/01 or later.
%
% This work has the LPPL maintenance status `maintained'.
% 
% The Current Maintainer of this work is Ivan Griffin
%
% This work consists of the files periodic_table.tex

%Description
%-----------
%periodic_table.tex - an example file illustrating the Periodic
%                     Table of Chemical Elements using TikZ

%Created 2009-12-08 by Ivan Griffin.  Last updated: 2010-01-11
%
%Thanks to Jerome
%-------------------------------------------------------------

\documentclass[border=1pt,tikz,landscape]{standalone}
\usepackage{tikz}
\usetikzlibrary{shapes,calc}
\usepackage[T1]{fontenc}
\usepackage[utf8]{inputenc}
\usepackage{lmodern}
\usepackage{color}
\usepackage[pdftex]{graphicx}
\usepackage[icelandic]{babel}
\usepackage{multirow}
\usepackage[separate-uncertainty=true,multi-part-units=single,output-decimal-marker={,},group-separator = {.}]{siunitx}
\addto\extrasicelandic{\sisetup{locale = DE}}

\newcommand{\CommonElementTextFormat}[4]
{
  \begin{minipage}{2.2cm}
    \centering
      {\textbf{#1} \hfill #2}%
      \linebreak \linebreak
      {\textbf{#3}}%
      \linebreak \linebreak
      {{#4}}
  \end{minipage}
}

\newcommand{\NaturalElementTextFormat}[4]
{
  \CommonElementTextFormat{#1}{#2}{\LARGE {#3}}{#4}
}

\newcommand{\OutlineText}[1]
{
\ifpdf
  % Couldn't find a nicer way of doing an outline font with TikZ
  % other than using pdfliteral 1 Tr
  %
  \pdfliteral direct {0.5 w 1 Tr}{#1}%
  \pdfliteral direct {1 w 0 Tr}%
\else
  % pstricks can do this with \pscharpath from pstricks
  %
  \pscharpath[shadow=false,
    fillstyle=solid,
    fillcolor=white,
    linestyle=solid,
    linecolor=black,
    linewidth=.2pt]{#1} 
\fi
}

\newcommand{\SyntheticElementTextFormat}[4]
{
\ifpdf
  \CommonElementTextFormat{#1}{#2}{\OutlineText{\LARGE #3}}{#4}
\else
  % pstricks approach results in slightly larger box
  % that doesn't break, so fudge here
  \CommonElementTextFormat{#1}{#2}{\OutlineText{\Large #3}}{#4}
\fi
}


\begin{document}


\begin{tikzpicture}[font=\sffamily, scale=0.5, transform shape]

%% Element Styles
  \tikzstyle{Element} = [draw=black, ElementFill,
    minimum width=2.75cm, minimum height=2.75cm, node distance=2.75cm]
  \tikzstyle{AlkaliMetal} = [Element, AlkaliMetalFill]
  \tikzstyle{AlkalineEarthMetal} = [Element, AlkalineEarthMetalFill]
  \tikzstyle{Metal} = [Element, MetalFill]
  \tikzstyle{Metalloid} = [Element, MetalloidFill]
  \tikzstyle{Nonmetal} = [Element, NonmetalFill]
  \tikzstyle{Halogen} = [Element, HalogenFill]
  \tikzstyle{NobleGas} = [Element, NobleGasFill]
  \tikzstyle{LanthanideActinide} = [Element, LanthanideActinideFill]
  \tikzstyle{PeriodLabel} = [font={\sffamily\LARGE}, node distance=2.0cm]
  \tikzstyle{GroupLabel} = [font={\sffamily\LARGE}, minimum width=2.75cm, node distance=2.0cm]
  \tikzstyle{TitleLabel} = [font={\sffamily\Huge\bfseries}]


  \node at ($(La.north -| Fr.west) + (5em,-15em)$) [name=elementLegend, Element, fill=white]
    {\NaturalElementTextFormat{}{}{Rafeind}{}};
\node at ($(La.north -| Fr.west) + (15em,-15em)$) [name=elementLegend, Element, fill=white]
    {\NaturalElementTextFormat{}{}{Róteind}{}};
\node at ($(La.north -| Fr.west) + (25em,-15em)$) [name=elementLegend, Element, fill=white]
    {\NaturalElementTextFormat{}{}{Nifteind}{}};

\end{tikzpicture}

\end{document}
