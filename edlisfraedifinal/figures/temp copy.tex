%----------------------------------------------------------------------
% Use TikZ/PGF to programmatically draw spacetime diagrams for 
% uniformly accelerated observers. Set the acceleration, initial
% conditions, and other paramters below.
% 
% Questions/Comments to Robert McNees at rmcnees@luc.edu
% http://jacobi.luc.edu
% @mcnees on Twitter
% January 2015, Updated October 2016
%----------------------------------------------------------------------

 
\documentclass[border=1mm, tikz]{standalone}
\usepackage{amsmath}
\usepackage{amsfonts}
\usepackage{mathrsfs}

\usepackage{tikz}
\usetikzlibrary{arrows.meta,decorations.pathmorphing,math}


%-----------------------------------------------------------------------
% Custom colors used in links and graphics
%-----------------------------------------------------------------------
\definecolor{plum}{rgb}{0.36078, 0.20784, 0.4}
\definecolor{chameleon}{rgb}{0.30588, 0.60392, 0.023529}
\definecolor{cornflower}{rgb}{0.12549, 0.29020, 0.52941}
\definecolor{scarlet}{rgb}{0.937, 0.161, 0.161}
\definecolor{brick}{rgb}{0.64314, 0, 0}
\definecolor{sunrise}{rgb}{0.80784, 0.36078, 0}

\newcommand{\scri}{\mathscr{I}}

\begin{document}
		
\pagestyle{empty} 

%-----------------------------------------------------------------------
% Set constants (speed of light, acceleration) 
%-----------------------------------------------------------------------
% The speed of light, c=1
\newcommand*{\sol}{1.0}
% The acceleration a
\newcommand*{\accel}{0.5}%

%-----------------------------------------------------------------------
% Set initial values (initial t, x, and v)
%-----------------------------------------------------------------------
% The initial time t_0
\newcommand*{\tinit}{-0.0}%
% The initial position x_0
\newcommand*{\xinit}{0.4}%
% The initial velocity. Obviously the magnitude must be less than \sol!
\newcommand*{\vinit}{0.0}%

%-----------------------------------------------------------------------
% Set the maximum value of t
%-----------------------------------------------------------------------
\newcommand*{\tmax}{5.0}%

%-----------------------------------------------------------------------
% Set the number of time intervals
%-----------------------------------------------------------------------
\newcommand*{\numintervals}{4}

%-----------------------------------------------------------------------
% Calculate the intervals \Delta t.
%-----------------------------------------------------------------------
\pgfmathsetmacro{\tinterval}{divide(\tmax-\tinit,\numintervals+1)}

%-----------------------------------------------------------------------
% Calculate the initial proper velocity.
%-----------------------------------------------------------------------
\pgfmathsetmacro{\uinit}{divide(\vinit,sqrt(1-pow(divide(\vinit,\sol),2)))}%

%-----------------------------------------------------------------------
% Calculate the initial value of the relativistic parameter gamma.
%-----------------------------------------------------------------------
\pgfmathsetmacro{\gammainit}{divide(1,sqrt(1-divide(pow(\vinit,2),1)))}%

%-----------------------------------------------------------------------
% Now define the position x(t) of the accelerated observer. 
%-----------------------------------------------------------------------
\pgfmathdeclarefunction{pos}{1}{%
  \pgfmathparse{divide(pow(\sol,2),\accel)*(sqrt(1+pow(divide(\accel,\sol)*(#1-\tinit) + divide(\uinit,\sol),2)) - sqrt(1 + pow(divide(\uinit,\sol),2)) ) + \xinit}
}

%-----------------------------------------------------------------------
% Points on the observer's worldline are a constant proper distance 
% from the point (\xstar,\tstar).
%-----------------------------------------------------------------------
% Define \xstar
\pgfmathsetmacro{\xstar}{\xinit - divide(pow(\sol,2),\accel)*sqrt(1+pow(divide(\uinit,\sol),2))}%
% Define \tstar
\pgfmathsetmacro{\tstar}{\tinit - divide(\vinit,\accel)*\gammainit}%

%-----------------------------------------------------------------------
% Find the maximum value of x(t) so we know how big the plot should be.
%-----------------------------------------------------------------------
\pgfmathsetmacro{\xmax}{pos(\tmax)}%

%-----------------------------------------------------------------------
% Draw everything
%-----------------------------------------------------------------------
\begin{tikzpicture}[>=LaTeX]
	  %-----------------------------------------------------------------------
	  % Clip a rectangular region so we can draw things later without 
	  % worrying about them poking out the sides of the diagram.	
	  %-----------------------------------------------------------------------
	  % First, set a lower value of t depending on whether \tinit is positive or negative.
	  \tikzmath{
		  if \tinit>0.0 then {let \tlower = -0.5*\tmax;} else {let \tlower = \tinit-0.5*\tmax;};
	  }
	  % Now clip the region. The +0.1 and +0.3 add padding for x and t labels and make sure the clip region doesn't end on a grid line.
	  \clip (-\xmax+0.1,\tlower) -- (\xmax+0.3,\tlower) -- (\xmax+0.3,\tmax+0.3) --  (-\xmax+0.1,\tmax+0.3) -- (-\xmax+0.1,\tlower);

	  %-----------------------------------------------------------------------
	  % Draw a background grid.
	  %-----------------------------------------------------------------------
	  \draw[step=.25,cornflower!15] (-\xmax,-\tmax) grid (\xmax+0.3,\tmax+0.3);
	  \draw[step=1.0,cornflower!45] (-\xmax,-\tmax) grid (\xmax+0.3,\tmax+0.3);

	  %-----------------------------------------------------------------------
	  % Draw the x and t axes.
	  %-----------------------------------------------------------------------
  	  \draw [-{Latex[]},semithick] (-\xmax,0) -- (\xmax,0) node[yshift=0.75em, fill=white, inner sep=0.2ex] {\large$x$};
  	  \draw [-{Latex[]},semithick] (0,-\tmax) -- (0,\tmax) node[xshift=0.75em, fill=white, inner sep=0.2ex] {\large $t$};

	  %-----------------------------------------------------------------------
	  % Draw lines from (\xstar,\tstar) to \numintervals evenly-spaced 
	  % points on the worldline.
	  %-----------------------------------------------------------------------
	  \foreach \x in {0,1,...,\numintervals} {
	    \draw[cornflower] (\xstar,\tstar) -- ({pos(\tinit+\x*\tinterval)},{\tinit+\x*\tinterval});
	  }

	  %-----------------------------------------------------------------------
	  % Plot the worldline of the observer as a smooth curve with no
	  % arrows on it.
	  %-----------------------------------------------------------------------
	  % \draw[thick, chameleon, domain=\tinit:\tmax,smooth,variable=\t,->]
      % 	 plot[] ({pos(\t)},{\t});
	
	  %-----------------------------------------------------------------------
	  % Plot the worldline as \numintervals equally spaced (in t)
	  % segments with an arrow on each one.
	  %-----------------------------------------------------------------------
	  \foreach \i in {0,1,...,\numintervals} {
		  \pgfmathsetmacro{\tstart}{\tinit+\i*\tinterval-0.05};
		  \pgfmathsetmacro{\tfin}{\tinit+(\i+1)*\tinterval};
		  \draw[thick, chameleon, domain=\tstart:\tfin,smooth,variable=\t,->]
			plot[] ({pos(\t)},{\t});
  	  }
 

	  %-----------------------------------------------------------------------
	  % Make a small, filled circle at (\xinit,\tinit).
	  %-----------------------------------------------------------------------
	  \fill[chameleon,opacity=1] (\xinit,\tinit) circle (2pt);
	  % Optionally add a label next to the point.
	  %\node at (\xinit+.75,\tinit) {\small $(x_0,t_0)$};

	  %-----------------------------------------------------------------------
	  % Draw a small, filled circle at (\xstar, \tstar).
	  %-----------------------------------------------------------------------
	  \fill[scarlet] (\xstar,\tstar) circle (2pt);

	  %-----------------------------------------------------------------------
	  %Draw the light cone of (\xstar,\tstar).
	  %-----------------------------------------------------------------------
	  \draw[scarlet, thick, dashed, domain=-\tmax:\tmax,smooth,variable=\t] 
	  	plot ({\sol*\t+\xstar},{\t+\tstar});
	  \draw[scarlet, thick, dashed, domain=-\tmax:\tmax,smooth,variable=\t] 
	  	plot ({-\sol*\t+\xstar},{\t+\tstar});
	  
\end{tikzpicture}   
  
\end{document}