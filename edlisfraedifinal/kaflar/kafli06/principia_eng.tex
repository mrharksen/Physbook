\subsection*{AXIOMS, OR THE LAWS OF MOTION}

\begin{enumerate}[label = \textbf{Law \arabic*} \hspace{0.3cm}]
    \item \textit{Every body perseveres in its state of being at rest or of moving uniformly straight forward, except insofar as it is compelled to change its state by forces impressed.}
    \vspace{0.25cm}
    
    \hspace{0.3cm} Projectiles persevere in their motions, except insofar as they are retarded by the resistance of the air and are impelled downward by the force of gravity. A spinning hoop, which has parts that by their cohesion continually draw one anohter back from rectilinear motions, does not cease to rotate, except insofar as it is retarded by the air. And larger bodies - planets and comets - preserve for a longer time both their progressive and their circular motions, which take place in spaces having less resistance.
    \item \textit{A change in motion is proportional to the motive force impressed and takes place along the straight line in which that force is impressed.} 
     \vspace{0.25cm}
    
    \hspace{0.3cm} If some force generates any motion, twice the force will generate twice the motion, and three times the force wil generate three times the motion, whether the force is impressed all at once or successively by degrees. And if the body was previously moving, the new motion (since motion is always in the same direction as the generative force) is added to the original motion if that motion was in the same direction or is subtracted from the original motion if it was in the opposite direction or, if it was in an oblique direction, is combined obliquely and compounded with it according to the directions of both motions.
    \item \textit{To any action there is always an opposite and equal reaction; in other words, the actions of two bodies upon each other are always equal and always opposite in direction.}
     \vspace{0.25cm}
    
    \hspace{0.3cm} Whatever presses or draws something else is pressed or drawn just as much by it. If anyone presses a stone with a finger, the finger is also pressed by the stone. If a horse draws a stone tied to a rope, the horse will (so to speak) also be drawn back equally toward the stone, for the rope, stretched out at both ends, will urge the horse toward the horse by one and the same endeavor to go slack and will impede the forward motion of the one as much as it promotes the forward motion of the other. If some body impinging upon another body changes the motion of that body in any way by its own force, then, by the force of the other body (because of the equality of their mutual pressure), it also will in turn undergo the same change in its own motion in the opposite direction. By means of these actions, equal changes occur in the motions, not in the velocities - that is, of course, if the bodies are not impeded by anything else. For the changes in velocities that likewise occur in opposite directions are inversely proportional to the bodies because the motions are changed equally. This law is valid also for attractions, as will be proved in the next scholium.
\end{enumerate}