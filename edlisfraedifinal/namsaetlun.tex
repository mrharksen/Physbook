\chapter*{Námsáætlun}

\section*{Eðlisfræði fyrir 6.~bekk eðlisfræðideildar II}

\begin{tcolorbox}
\begin{table}[H]
    \begin{tabular}{l l}
       \textbf{Námskeið:} EÐLI3SK07/EÐLI2RF06 & \textbf{Kennari:} Matthias B. Harksen \emph{(\href{mailto:matthias@mr.is}{matthias{\fontfamily{ptm}\selectfont @}mr.is})}
    \end{tabular}
\end{table}
\vspace{-0.5cm}
\begin{table}[H]
    \begin{tabular}{l l}
    \textbf{Kennslubók:} Physics for Scientists and Engineers - Randall D. Knight \emph{(ISBN: 978-1-292-15742-9)} &
    \end{tabular}
\end{table}
\end{tcolorbox}

\section*{Viðmið}

\subsection*{Þekkingarviðmið haustannar:}

Gert er ráð fyrir að nemandinn hafi öðlast þekkingu og skilning á eftirfarandi hugtökum:

\begin{itemize}
    \item Hleðslur, rafkraftar, rafsvið, straumur, spenna og viðnám.
    \item Lögmál Gauss um rafsvið.
    \item Þéttar, rýmd, samband spennu og rafsviðs, jafnspennurásir með viðnámum og þéttum.
    \item Segulsvið, lögmál Gauss fyrir segulsvið, lögmál Biot-Savart og lögmál Ampéres.
    \item Kraftur á straumleiðara, kraftur á ögn í segulsviði, spanlögmál Faradays, spólur og sjálfspan.
    \item Riðspennurásir með raðtengdu viðnámi, þétti og spólu.
    \item Jöfnum Maxwells á tegurformi.
\end{itemize}


\subsection*{Þekkingarviðmið vorannar:}

Gert er ráð fyrir að nemandinn hafi öðlast þekkingu og skilning á eftirfarandi hugtökum:

\begin{itemize}
    \item Grunnhugtök takmörkuðu afstæðiskenningarinnar, rafsegulbylgjur,  tregðukerfi, Galíleójöfnunurnar, afstæðislögmál Einsteins, Lorentz-ummyndanir.
    \item Skilgreiningu skriðþunga sem er varðveittur afstætt, massa sem orkuformi og skilgreiningu afstæðrar hreyfiorku.
    \item Jafngildislögmál Einsteins.
    \item Þróunarsaga skammtafræðinnar svo sem skammtakenning Plancks, ljóseindakenning Einsteins, röntgengeislun, atómlíkan Rutherfords, atómlíkan Bohrs, skömmtun orku í vetnisatómi, agnabylgjur de Broglie.
    \item Grunnatriði í kjarneðlisfræði: kjarnakraftar, bindiorka kjarnans og geislun, kjarnorka.
\end{itemize}


\newpage

\section*{Haustönn}

\begin{table}[H]
    \centering
    \begin{tabular}{|c|c|l|c|c|}
    \hline
       \textbf{Vika}  & \textbf{Dagsetning} & \textbf{Efnistök} & \textbf{Kafli} & \textbf{Undirgreinar}  \\ \hline \hline
        $1$ & 20.08-27.08 & Rafhleðslur og lögmál Coulombs & 22 & 1-5 \\ \hline
        $2$ & 30.08-03.09 &  Rafsvið og tvískautsvægi & 23 & 1-2, 5-7 \\ \hline
        $3$ & 06.09-10.09 & Lögmál Gauss & 24 & 1-6 \\ \hline
        $4$ & 13.09-17.09 & Rafspenna & 25 & 1-2, 4-6 \\ \hline
        $5$ & 20.09-24.09 & Tengsl rafspennu og rafsviðs & 26 & 1-7 \\ \hline
        $6$ & 27.09-01.10 & Rafstraumur og viðnám & 27 & 1-5 \\ \hline
        $7$ & 04.10-08.10 & Rafrásir & 28 & 1-5 \\ \hline
        $\mathbf{8}$ & 11.10-14.10 & Rafrásir & 28 & 6-9 \\ \hline
        $\mathbf{9}$ & 19.10-22.10 & Segulsvið & 29 & 1-5 \\ \hline
        $10$ & 25.10-29.10 & Segulsvið & 29 & 6-10 \\ \hline
        $11$ & 01.11-05.11 & Span & 30 & 1-5 \\ \hline
        $12$ & 08.11-12.11 & Span  & 30  & 7-9 \\ \hline
        $13$ & 15.11-19.11 & Rafsegulbylgjur & 31 & 2-4 \\ \hline
        $14$ & 22.11-26.11 & Riðstraumsrásir & 32 & 1-2, 4-6 \\ \hline
    \end{tabular}
\end{table}

\section*{Vorönn}

\begin{table}[H]
    \centering
    \begin{tabular}{|c|c|l|c|c|}
    \hline
       \textbf{Vika}  & \textbf{Dagsetning} & \textbf{Efnistök} & \textbf{Kafli} & \textbf{Undirgreinar}  \\ \hline \hline
        $1$ & 04.01-07.01 & Bylgjueðli ljóss & 33 & 1-3 \\ \hline
        $2$ & 10.01-14.01  &  Bylgjueðli ljóss & 33 & 1-3, 8 \\ \hline
        $3$ & 17.01-21.01 & Ljósgeislafræði & 34 & 1-4 \\ \hline
        $4$ & 24.01-28.01 & Ljósgeislafræði & 34 & 5-6 \\ \hline
        $5$ & 31.01-04.02 & Linsur & 35 & 4-5 \\ \hline
        $6$ & 07.02-11.02 & Afstæðiskenningin & 36 & 1-4 \\ \hline
         $\mathbf{7}$ & 14.02-16.02 & Afstæðiskenningin & 36 & 5-7 \\ \hline
        $8$ & 21.02-25.02 & Afstæðiskenningin & 36 & 8-10 \\ \hline
        $9$ & 28.02-04.03 & Undirstöður nútímaeðlisfræði & 37 & 1-5 \\ \hline
        $10$ & 07.03-11.03 & Undirstöður nútímaeðlisfræði & 37 & 6-8 \\ \hline
        $11$ & 14.03-18.03 & Inngangur að skammtafræði & 38 & 1-4 \\ \hline
        $12$ & 21.03-25.03 & Inngangur að skammtafræði  & 38  & 5-7 \\ \hline
        $13$ & 28.03-01.04  & Kjarneðlisfræði & 42 & 1-3\\ \hline
        $14$ & 04.04-08.04 & Kjarneðlisfræði & 42 & 5-6 \\ \hline
    \end{tabular}
\end{table}

\newpage

\section*{Námsmat}

\subsubsection*{EÐLI3SK07 (Kennileg eðlisfræði)}

Hér byggir námseinkunin á eftirfarandi þáttum:

\begin{table}[H]
    \centering
    \begin{tabular}{|c|l|}
    \hline
        Hlutfall & \hspace{0.75cm}Þáttur \\ \hline \hline
        $20 \%$ & Jólapróf \\ \hline
        $60 \%$ & Skriflegar æfingar \\ \hline
        $20 \%$ & Mat kennara\\ \hline
    \end{tabular}
\end{table}

Þar sem að:

\begin{itemize}
    \item Jólaprófið er 90 mínútna próf úr námsefni haustannar í 6.~bekk (rafsegulfræði).
    
    \item Haldnar verða a.m.k.~4 skriflegar æfingar á hvorri önn. Þrjár bestu æfingarnar á hvorri önn gilda.
    
    \item Mat kennara verður metið út frá frammistöðu nemandans við töflureikning í tímum, vinnubrögðum nemandans við heimanám, skriflegri framsetningu á eðlisfræði og eðlisfræðilegu innsæi.
\end{itemize}

\subsubsection*{EÐLI2RF06 (Verkleg eðlisfræði)}

Hér byggir námseinkunin á eftirfarandi þáttum:

\begin{table}[H]
    \centering
    \begin{tabular}{|c|l|}
    \hline
        Hlutfall & \hspace{0.5cm}Þáttur \\ \hline \hline
        $50\%$ & Skýrslur \\ \hline
        $30\%$ & Verkbók \\ \hline
        $20\%$ & Mat kennara \\ \hline
    \end{tabular}
\end{table}

Þar sem að:

\begin{itemize}
    \item Þrjár tímaskýrslur verða haldnar yfir skólaárið. Tvær bestu skýrslurnar gilda.
    
    \item Verkbók nemenda verður metin eftir hverja lotu (þrjár lotur).
    
    \item Mat kennara verður metið út frá mætingu og frammistöðu nemandans í verklegum tímum, vinnubrögðum nemandans við úrvinnslu verklegra tilrauna, skriflegri framsetningu á töflum, gröfum og mæliniðurstöðum ásamt verklegu innsæi.
\end{itemize}


\begin{tcolorbox}
\renewcommand{\thempfootnote}{\arabic{mpfootnote}}
\textbf{7.2.}~\iduck{Einkunnir á stúdentsprófum eru gefnar í heilum og hálfum tölum frá $1$ til $10$.  Aðaleinkunn á stúdentsprófi er reiknuð þannig: Reiknað er vegið meðaltal námseinkunna og vegið meðaltal prófseinkunna.  Meðaltal þessara tveggja talna er aðaleinkunn á stúdentsprófi. Vægi einkunna fer eftir heildareiningafjölda í greininni öll námsárin þrjú. Ef gefnar eru fleiri en ein einkunn í grein skiptist einingafjöldinn milli þeirra. Námseinkunn er gefin að vori í öllum stúdentsprófsgreinum og er þar tekið tillit til einkunna nemenda á jólaprófi og frammistöðu og ástundunar um veturinn.}

\vspace{0.2cm}

\textbf{3.1.}~\iduck{Þátttaka nemenda í verkefnum og æfingum er ófrávíkjanleg skylda og ber kennara að gefa einkunnina 0 þeim sem koma ekki í slíkar æfingar eða skila ekki verkefnum nema um sannanleg veikindi eða önnur óhjákvæmileg forföll sé að ræða. Slík veikindi eða forföll þarf að tilkynna áður en æfing er haldin eða verkefni skilað.}\hspace{0.05cm}

\vspace{0.2cm}

\textbf{6.1.}~\iduck{Gert er ráð fyrir að nemendur mæti í alla verklega tíma og skili verkefnum í tengslum við þá. Um vægi þessa námsþáttar má sjá í námsáætlun viðkomandi námsgreinar. Ef nemandi sækir ekki a.m.k.~$75\%$ verklegra æfinga og/eða skilar ekki a.m.k.~$75\%$ skýrslna fær hann $0$ í verklegum þætti námseinkunnar.}\hspace{0.025cm} \footnote{\href{https://mr.is/skolinn/skolareglur/skolasoknarreglur/}{Skólasóknarreglur}}
\end{tcolorbox}

\newpage

\subsection*{Stúdentspróf}


Haldin verða tvö þriggja tíma stúdentspróf í hvoru námskeiði (EÐLI3SK07/EÐLI2RF06).

\subsubsection*{Stúdentspróf í Eðlisfræði I (EÐLI3SK07)}

Prófað verður úr námsefni fyrir jól í 5.~bekk og fyrir jól í 6.~bekk. (Kaflar 1-11 og 22-32 í Randall Knight)

Skiptingin verður í grófum dráttum eftirfarandi:

\begin{table}[H]
    \centering
    \begin{tabular}{|c|l|}
    \hline
        Hlutfall & \hspace{0.75cm}Þáttur \\ \hline \hline
        $10 \%$ & Fræðilegar spurningar/útleiðslur úr námsefni 5.~bekkjar. \\ \hline
        $20 \%$ & Fræðilegar spurningar/útleiðslur úr námsefni 6.~bekkjar. \\ \hline
        $20 \%$ & Dæmareikningur úr námsefni 5.~bekkjar \\ \hline
        $50 \%$ & Dæmareikningur úr námsefni 6.~bekkjar \\ \hline
    \end{tabular}
\end{table}


Efnistökin sem prófað verður úr eru eftirfarandi (þetta er ekki endanlegur listi):

\begin{itemize}
    \item \textbf{Gangfræði:} Staða, hraði, hröðun og stöðujöfnur fyrir fasta hröðun.
    \item \textbf{Kraftar:} Kraftamyndir, lögmál Newtons, þverkraftar, togkraftar, núningskraftar.
    \item \textbf{Varðveislulögmálin:} Hreyfiorka, stöðuorka, heildarorka, vinna, afl, skriðþungi, vinnulögmálið.
    \item \textbf{Rafstöðufræði:} Lögmál Coulombs, Rafsvið, lögmál Gauss, rafspenna, þéttir, rýmd.
    \item \textbf{Rafrásir:} Rafstraumur, lögmál Kirchoffs, viðnám, eðlisviðnám, hliðtenging, raðtenging, riðstraumur.
    \item \textbf{Rafsegulfræði:} Segulsvið, segulflæði, lögmál Biot-Savart; Amperes; Faradays og Lenz. 
\end{itemize}



\subsubsection*{Stúdentspróf í Eðlisfræði II (EÐLI2RF06)}

Prófað verður úr námsefni eftir jól í 5.~bekk og 6.~bekk. (Kaflar 12-21, 33-38 og 42 í Randall Knight). Þar að auki verður prófað úr verklegum æfingum sem haldnar hafa verið yfir bæði árin.

Skiptingin verður í grófum dráttum eftirfarandi:

\begin{table}[H]
    \centering
    \begin{tabular}{|c|l|}
    \hline
        Hlutfall & \hspace{0.75cm}Þáttur \\ \hline \hline
        $15 \%$ & Verkleg úrvinnsla úr tilraunum 5.~og 6.~bekkjar: Óvissureikningar, aðhvarfsgreining og gröf. \\ \hline
        $20 \%$ & Fræðilegar spurningar/útleiðslur úr námsefni 5.~og 6.~bekkjar. \\ \hline
        $20 \%$ & Dæmareikningur úr námsefni 5.~bekkjar \\ \hline
        $40 \%$ & Dæmareikningur úr námsefni 6.~bekkjar \\ \hline
    \end{tabular}
\end{table}



Efnistökin sem prófað verður úr eru eftirfarandi (þetta er ekki endanlegur listi):

\begin{itemize}
    \item \textbf{Snúningar:} Horn, hornhraði, hornhröðun, hverfitregða, hverfiþungi, kraftvægi.
    \item \textbf{Þyngdarlögmálið:} Þyngdarlögmálskrafturinn, stöðuorka þyngdarlögmálskraftsins og lögmál Keplers.
    \item \textbf{Vökvar:} Þrýstingur, lögmál Pascals; Bernoullis og Arkímedesar, vökvaflæði og samfeldnilögmálið.
    \item \textbf{Einföld sveifluhreyfing:} Sveiflutími, tíðni, sveiflutíðni, útslag, fasahorn og hreyfilýsing.
    \item \textbf{Bylgjur:} Bylgjulengd, bylgjuhraði, skynstyrkur, Doppler-hrif, bylgjusamliðun og eiginsveifluhættir.
    \item \textbf{Varmafræði:} Varmaorka, eðlisvarmi, gaslögmálið og hitastig.
    \item \textbf{Ljósgeislafræði:} Bylgjuvíxl, raufamynstur, lögmál Snells og linsujafnan.
    \item \textbf{Takmarkaða afstæðiskenningin:} Tímaslenging, lengdarstytting, hraðasamlagning, kyrrstöðuorka.
    \item \textbf{Inngangur að skammtafræði:} Bohr-líkanið, orkuskömmtun, stærð frumeindakjarnans.
    \item \textbf{Verklegt:} Óvissureikningar, aðhvarfsgreining og gröf.
\end{itemize}

\newpage

\section*{Leslisti fyrir jólapróf}

Eftirfarandi dæmatímar eru til prófs:

\begin{table}[H]
    \centering
    \begin{tabular}{|l|l|l|}
    \hline
    \textbf{Dæmatími} & \textbf{Efnistök} & \textbf{Dæmanúmer í RK}
     \\ \hline \hline
        Dæmatími 1 & Coulombskrafturinn & 22.13, 22.16, 22.17, 22.19  \\ \hline
        Dæmatími 2 & Rafsviðið & 22.26, 22.32, 22.63, 22.67 \\ \hline
        Dæmatími 3 & Rafkrafturinn & 22.5, 22.54, 22.73, 22.75 \\ \hline
        Dæmatími 4 & Rafflæði & 24.9, 24.11, 24.16, 24.29 \\ \hline
        Dæmatími 5 & Plötuþéttir & 23.3, 23.23, 23.25, 23.26 \\ \hline
        Dæmatími 6 & Hreyfing í rafsviði & 23.37, 23.52, 23.53, 23.73 \\ \hline
        Dæmatími 7 & Lögmál Gauss & 24.33, 24.43, 23.54, 24.55 \\ \hline
        Dæmatími 8 & Rafstöðuorka & 25.12, 25.13, 25.16, 25.20 \\ \hline
        Dæmatími 9 & Rafspenna &  25.30, 25.31, 25.32, 25.33 \\ \hline
        Dæmatími 10 & Þéttir & 25.22, 25.23, 25.25, 25.26 \\ \hline
        Dæmatími 11 & Rýmd þéttis og rafsvarar & 26.21, 26.23, 26.35, 26.36 \\ \hline
        Dæmatími 12 & Orkan í þétti &  26.31, 26.33, 26.64, 26.67 \\ \hline
        Dæmatími 13 & Jafngildir þéttar & 26.27, 26.28, 26.56, 26.57 \\ \hline
        Dæmatími 14 & Jafngild viðnám & 28.25, 28.26, 28.27, 28.28 \\ \hline
        Dæmatími 15 & Lögmál Kirchoffs & 28.4, 28.31, 28.52, 28.63   \\ \hline
        Dæmatími 16 & Afl í rafrásum & 28.7, 28.8, 28.9, 28.78 \\ \hline
        Dæmatími 17 & Eðlisviðnám & 27.27, 27.28, 27.33, 27.37 \\ \hline 
        Dæmatími 18 & Segulflæði og lögmál Ampéres  & 30.4, 30.5, 29.22, 29.23 \\ \hline
        Dæmatími 19 & Segulsvið umhverfis beinan vír & 29.8, 29.9, 29.14, 29.15 \\ \hline
        Dæmatími 20 & Lögmál Biot-Savart og segulkrafturinn & 29.5, 29.6, 29.26, 29.27   \\ \hline
        Dæmatími 21 & Hringhreyfing í segulsviði  & 29.30, 29.31, 29.63, 29.65 \\ \hline 
        Dæmatími 22 & Segulkrafturinn á vír &  29.36, 29.34, 29.72, 29.73  \\ \hline
        Dæmatími 23 & Segulsvið langspólu & 29.25, 29.49, 29.50, 29.79  \\ \hline
        Dæmatími 24 & Lögmál Faradays og lögmál Lenz & 30.11, 30.13, 30.15, 30.14  \\ \hline
        Dæmatími 25 & Spanstraumur & 30.53, 30.54, 30.55, 30.59 \\ \hline
        Dæmatími 26 & Spanspólur og orkan í segulsviði & 30.12, 30.26, 30.27, 30.28 \\ \hline
        Dæmatími 27 & RC-rásir  & 28.36, 28.68, 28.70, 28.80  \\ \hline
        Dæmatími 28 & LR-rásir  & 30.16, 30.34, 30.35, 30.79 \\ \hline
        Dæmatími 29 & LC-rásir & 30.71, 30.72, 30.73, 30.33 \\ \hline
        Dæmatími 30 & Riðspenna: RCL-rás & 32.30, 32.32, 32.52, 32.53 \\ \hline
    \end{tabular}
\end{table}


Eftirfarandi útleiðslur (og tilheyrandi skilgreiningar) eru til prófs:


\begin{multicols}{3}
\begin{itemize}
    \item Rafsvið punkthleðslu
    \item Rafsviðið frá einni plötu
    \item Rafsviðið í plötuþétti
    \item Rýmd þéttis
    \item Orkan í þétti
    \item Raðtenging viðnáma
    \item Hliðtenging viðnáma
    \item Raðtenging þétta
    \item Hliðtenging þétta
    \item Raðtenging spóla
    \item Hliðtenging spóla
    \item Segulsvið umhverfis vír
    \item Segulsvið langspólu
    \item Hringhraðaltíðni í segulsviði
    \item Spanstuðull spólu
    \item Orkan í spólu
    \item Afhleðsla þéttis
    \item Hrörnunarhreyfing LR-rásar
    \item Sveifluhreyfing LC-rásar
    \item Rafsvið meðfram samhverfuás hringlaga gjarðar
    \item Segulsvið í miðju hringlaga gjarðar
    \item Hreyfilýsing RCL-rásar
    \item Framsetning á Maxwells jöfnunum
\end{itemize}

\end{multicols}


\newpage




\section*{Verklegt: Lota 1}

\begin{multicols}{2}
\begin{enumerate}[label = \textbf{Mán \arabic*:}]
    \item Eva Mítra, Ingi Hrafn
    \item Eðvald Þór, Egill Grétar
    \item Aron, Mónika
    \item Bjarki, Oddur
    \item Helgi Níels, Magnús Nói
    \item Hannes, Elísa Inger
    \item Eiður, Davíð Þór, Davíð Freyr \columnbreak
\end{enumerate}

\begin{enumerate}[label = \textbf{Fös \arabic*:}]
    \item Sigurður Ari, Jökull
    \item Teresa Ann, Kristín Ingibjörg
    \item Gunnar Snorri, Þorsteinn Wilhelm
    \item Ragnar Björn, Ólafur Björn
    \item Kristján Dagur, Hildur, Karitas Telma
\end{enumerate}
\end{multicols}

\vspace{0.5cm}

\begin{multicols}{2}
\begin{enumerate}[label = \textbf{Þri \arabic*:}]
    \item Einar Andri, Ingibjörg Brynja
    \item Benedikt Kristinn, Sandra
    \item Einar Atli, Hjörtur Viðar
    \item Guðmundur Tómas, Ólafur Breki \columnbreak
\end{enumerate}

\begin{enumerate}[label = \textbf{Fim \arabic*:}]
    \item Þóra Dís, Ylja Karen
    \item Stefán Árni, Jovan
    \item Einar Skúli, Hulda Eir
    \item Haraldur, Alex Orri
\end{enumerate}
\end{multicols}

\hspace{0.5cm}

\begin{table}[H]
    \centering
    \begin{tabular}{|l||c|c|c|c|c|c|c|c|}
    \hline 
         & 23-27.08 & 30-3.09 & 06-10.09 & 13-17.09 & 20-24.09 & 27-01.10 & 4-8.10    \\ \hline \hline
        Hitastigull eirviðnáms & 1 &  & 6 & 5 & 4 & 3 & 2  \\ \hline
        Straumur, spenna og viðnám & 2 & 1 &  & 6 & 4 & 4 & 3 \\ \hline
        Íspenna rafhlöðu & 3 & 2 & 1 &  & 6 & 5 & 4 \\ \hline
        Stöðurafmagn & 4 & 3 & 2 & 1 &  & 6 & 5 \\ \hline
        Lögmál Coulombs & 5 & 4 & 3 & 2 & 1 &  & 6 \\ \hline
        Plötuþéttir & 6 & 5 & 4 & 3 & 2 & 1 & \\ \hline
        Sveiflusjá &  & 6 & 5 & 4 & 3 & 2 & 1 \\ \hline
    \end{tabular}
\end{table}


\begin{itemize}
    \item Það verða upptektartímar í árshátíðarvikunni: 11-14.~október.
    \item 6.$X_2$ verður með tímaskýrslu fimmtudaginn 21.~október
    \item 6.Y verður með tímaskýrslu föstudaginn 22.~október.
\end{itemize}

Tímaskýrslan verður með eftirfarandi sniði: Þið megið einungis mæta með verkbókina ykkar, skriffæri og reiknivél (það er því ekki í lagi að vera með verkbók á iPad formi). Þið dragið tvær af tilraunum lotunnar og veljið þá sem að þið viljið frekar skrifa um. Skýrslan er skrifuð á A3 örk (svolítið eins og í íslenskum stíl). Öll gröf sem eiga að fylgja með eru handteiknuð á millimetrapappír (eins og á stúdentsprófinu í eðlisfræði).


\newpage


\section*{Verklegt: Lota 2}

\begin{multicols}{2}
\begin{enumerate}[label = \textbf{Mán \arabic*:}]
    \item Magnús Nói og Bjarki Þór
    \item Davíð Þór, Eðvald Þór og Mónika Andjani
    \item Davíð Freyr og Aron Dimas
    \item Egill Grétar og Eva Mítra
    \item Eiður og Ingi Hrafn
    \item Elísa Inger og Oddur \columnbreak
\end{enumerate}

\begin{enumerate}[label = \textbf{Fös \arabic*:}]
    \item Hannes og Teresa Ann
    \item Kristján Dagur og Helgi Níels
    \item Hildur og Jökull
    \item Ragnar Björn og Þorsteinn
    \item Karitas Telma og Sigurður Ari
    \item Gunnar Snorri, Kristín og Ólafur Björn
\end{enumerate}
\end{multicols}

\vspace{0.5cm}

\begin{multicols}{2}
\begin{enumerate}[label = \textbf{Þri \arabic*:}]
    \item Sædís Ósk og Benedikt Kristinn
    \item Einar Andri og Sandra
    \item Hjörtur og Ólafur Breki
    \item Ingibjörg Brynja, Guðmundur og Einar Atli \columnbreak
\end{enumerate}

\begin{enumerate}[label = \textbf{Fim \arabic*:}]
    \item Þóra Dís, Stefán Árni
    \item Einar Skúli, Haraldur
    \item Jovan, Hulda Eir
    \item Alex Orri, Ylja Karen.
\end{enumerate}
\end{multicols}

\hspace{0.5cm}

\begin{table}[H]
    \centering
    \begin{tabular}{|l||c|c|c|c|c|c|c|}
    \hline 
         & 1-5.11 & 8-12.11 & 15-19.11 & 22-26.11 & 4-7.1 & 10-14.1   \\ \hline \hline
        Langspóla & 1 & 6  & 5 & 4 & 3 & 2   \\ \hline
        Afhleðsla þéttis & 2 & 1 & 6 & 5 & 4 & 3 \\ \hline
        Segulsvið umhverfis beinan vír & 3 & 2 & 1 & 6 & 5 & 4 \\ \hline
        Span & 4 & 3 & 2 & 1 & 6  & 5  \\ \hline
        Ljósbrot & 5 & 4 & 3 & 2 & 1 & 6   \\ \hline
        Mæling á $\frac{e}{m}$ & 6 & 5 & 4 & 3 & 2 & 1  \\ \hline
    \end{tabular}
\end{table}


\begin{itemize}
    \item Það verða upptektartímar í vikunni: 17-21.~janúar.
    \item 6.$X_2$ verður með tímaskýrslu fimmtudaginn 27.~janúar.
    \item 6.Y verður með tímaskýrslu föstudaginn 28.~janúar.
\end{itemize}

Tímaskýrslan verður með eftirfarandi sniði: Þið megið einungis mæta með verkbókina ykkar, skriffæri og reiknivél (það er því ekki í lagi að vera með verkbók á iPad formi). Þið dragið tvær af tilraunum lotunnar og veljið þá sem að þið viljið frekar skrifa um. Skýrslan er skrifuð á hvítan A4 pappír. Öll gröf sem eiga að fylgja með eru handteiknuð á millimetrapappír (eins og á stúdentsprófinu í eðlisfræði). Þið skilið síðan verkbókinni ykkar að skýrslunni lokinni.

\newpage

\section*{Verklegt: Lota 3}

\begin{multicols}{2}
\begin{enumerate}[label = \textbf{Mán \arabic*:}]
    \item Elísa Inger og Eðvald Þór
    \item Bjarki Þór og Eva Mítra
    \item Ingi Hrafn og Davíð Freyr
    \item Mónika Andjani, Magnús Nói og Egill Grétar
    \item Aron Dimas  og Oddur og Eiður \columnbreak
\end{enumerate}

\begin{enumerate}[label = \textbf{Fös \arabic*:}]
    \item Kristján Dagur og Teresa Ann
    \item Jökull og Hannes
    \item Hildur, Þorsteinn Wilhelm  og Kristín Ingibjörg
    \item Ragnar, Gunnar Snorri og Ólafur Björn
    \item Karitas Telma og Helgi
\end{enumerate}
\end{multicols}

\vspace{0.5cm}

\begin{multicols}{2}
\begin{enumerate}[label = \textbf{Þri \arabic*:}]
    \item Guðmundur Tómas og Hjörtur Viðar
    \item Einar Atli og Benedikt Kristinn
    \item Ingibjörg Brynja og Sandra
    \item Einar Andri, Ólafur Breki og Sædís Ósk \columnbreak
\end{enumerate}

\begin{enumerate}[label = \textbf{Fim \arabic*:}]
    \item Alex Orri og Jovan Gajic
    \item Einar Skúli og Ylja Karen
    \item Þóra Dís og Hulda Eir
    \item Stefán Árni, Haraldur.
\end{enumerate}
\end{multicols}

\hspace{0.5cm}

\begin{table}[H]
    \centering
    \begin{tabular}{|l||c|c|c|c|c|c|c|}
    \hline 
         & 31-04.02 & 07-11.02 & 14-18.02 & 21-25.02 & 28-04.03 & 07-11.03   \\ \hline \hline
        Brennivídd safnlinsu & 1 & 6  & 5 & 4 & 3 & 2   \\ \hline
        Samliðun ljóss í raufagleri & 2 & 1 & 6 & 5 & 4 & 3 \\ \hline
        Litróf frumefna & 3 & 2 & 1 & 6 & 5 & 4 \\ \hline
        Geislavirkni & 4 & 3 & 2 & 1 & 6  & 5  \\ \hline
        Ákvörðun Plancksfasta & 5 & 4 & 3 & 2 & 1 & 6   \\ \hline
        Spóla í riðstraumsrás & 6 & 5 & 4 & 3 & 2 & 1  \\ \hline
    \end{tabular}
\end{table}


\begin{itemize}
    \item Það verða upptektartímar í vikunni: 14-18.~mars.
    \item 6.$X_2$ verður með tímaskýrslu fimmtudaginn 24.~mars.
    \item 6.Y verður með tímaskýrslu föstudaginn 25.~mars.
\end{itemize}

\newpage

\section*{Rannsóknarverkefni}

Nemendum stendur til boða að vinna rannsóknarverkefni sem gildir til upphækkunar um einn heilan á jólaprófi (einkunnin $x$ fyrir rannsóknarverkefnið gefur $0,x$ í hækkun á jólaprófi). Tímaáætlun er eftirfarandi:


\begin{table}[H]
    \centering
    \begin{tabular}{|l|l|}
    \hline
        Mánudaginn 4.~október & Skila \emph{Ágripi} að verkefninu.  \\ \hline
        Mánudaginn 31.~janúar & Skila lokaútgáfu af verkefninu. \\ \hline
        Föstudaginn 11.~febrúar & Skila glærum eða handriti að kynningu. \\ \hline
        Árshátíðarvikan 14-16. febrúar & Kynning á verkefni (nemendur valdir af handahófi). \\ \hline
    \end{tabular}
\end{table}

Hér fyrir neðan er listi af verkefnum sem að nemendur geta haft í huga þegar þeir eru að velja:

\begin{multicols}{3}

\begin{itemize}
    \item Tennisspaðasetningin (Dzhanibekov áhrifin)
    \item Staðallíkanið
    \item Þverganga Venusar
    \item Flóðkraftar
    \item Pólvelta jarðarinnar
    \item Hár Garðarbrúðu og Youngsstuðull þess
    \item Skammtímaferillinn (Brachistochrone)
    \item Slinky gormur að detta
    \item Maxwell djöfullinn og óreiða
    \item Eilfíðarvélar
    \item Hvers vegna sér maður hyllingar í eyðimörkum?
    \item Að meta stærð kjarnorkusprengjunnar eins og Fermi
    \item Aðferð Rømers til þess að mæla hraða ljóssins
    \item Fjarreikistjörnur
    \item Brownhreyfing
    \item Vagga Newtons
    \item Fallhlífarstökk
    \item Teygjustökk
    \item Corioliskrafturinn
    \item Einfalt líkan af stjörnu
    \item Hertzsprung-Russel línuritið
    \item Cavendish tilraunin
    \item Eratosthenes og mæling hans á geisla jarðarinnar
    \item Hohman brautarskipti
    \item Bell setningin
    \item EPR þversögnin
    \item Hvað eru skammtatölvur?
    \item Friedman-jöfnurnar
    \item Loftmótsstaða í Hvalfjarðargöngunum?
    \item Yoyo
    \item Leidenfrost áhrifin
    \item Euler-Lagrange jafnan
    \item Hvað ákvarðar hljóðið sem heyrist þegar bolti skoppar?
    \item Seigja vökva
    \item Strengjafræði
    \item Setning Noethers
    \item Runge-Lenz vigurinn
    \item Köttur Schrödingers
    \item Focault pendúllinn
    \item Jarðskjálftar
    \item Af hverju lenda kettir alltaf á fótunum?
    \item Hvernig virka norðurljós?
    \item Ofurflæðandi Helín
    \item Carnotrásin
    \item Magnus-krafturinn
    \item Olíudropatilraun Millikan
    \item Rutherford tvístrun
    \item Glundroði (e. chaos)
    \item Michelson-Morley tilraunin
\end{itemize}
\end{multicols}

Verkefnin geta flokkast niður í fjóra hópa:

\begin{enumerate}[label = \textbf{(\roman*)}]
    \item \textbf{Fræðileg umfjöllunarefni:} þar sem að nemandinn leiðir út niðurstöðu sem er ekki bein afleiðing af því námsefni sem að við höfum þegar tekið.
    \item \textbf{Líkanagerð og gagnagreining:} þar sem að nemandinn safnar að sér gögnum um umfjöllunarefnið og greinir þau (t.d.~með töflum eða gröfum).
    \item \textbf{Forritunarverkefni:} þar sem að nemandinn hefur útfært verkefni í forritunarmáli (helst Python). 
    \item \textbf{Greinargerð:} þar sem að nemandinn gerir grein fyrir einhverju tilteknu viðfangsefni á skýran og rökréttann hátt án þess þó endilega að nota neinar útleiðslur eða jöfnur til þess að skýra mál sitt.
\end{enumerate}


\newpage

\section*{Listi yfir rannsóknarefni nemenda}

\begin{table}[H]
    \centering
    \begin{tabular}{|l|l|}
    \hline  \textbf{Nafn} & \textbf{Efni} \\ \hline \hline
       Aron Dimas & Þrýstisuðupottar \\ \hline
       Bjarki Þór & Kaffibollaslysið  \\ \hline
        Davíð Freyr & Að drepa með blakbolta   \\ \hline
        Davíð Þór & Sólgos  \\ \hline
       Eðvald Þór & Af hverju lenda kettir alltaf á fótunum?  \\ \hline
       Egill Grétar & Eðlisfræðin í flugi \\ \hline
        Eiður & Fallbyssukúla með loftmótsstöðu og gormi \\ \hline
       Elísa Inger & Yoyo \\ \hline
        Eva Mítra & Kökubakstur og kjarnhitastig köku \\ \hline
       Gunnar Snorri & Er 5G skaðlegt? \\ \hline
       Hannes Ísberg & Hvernig léttist fólk?   \\ \hline
       Helgi Níels & Eilífðarvélar \\ \hline
      Hildur & Rutherford tvístrun   \\ \hline
       Ingi Hrafn & Hjólabretti  \\ \hline
       Jökull & Hillingar í eyðimörk  \\ \hline
       Karitas Telma & Michelson-Morley tilraunin   \\ \hline
       Kristín Ingibjörg & Ping-pong  \\ \hline
      Kristján Dagur & Tunglferðin   \\ \hline
       Magnús Nói & Magnus-krafturinn \\ \hline
       Mónika Andjani & Hrísgrjónapottur \\ \hline
       Oddur & Smiðjusteinn Skalla-Gríms \\ \hline
       Ólafur Björn & Þverganga Venusar \\ \hline
       Ragnar Björn & Styrkur í bardagaíþróttum  \\ \hline
       Sigurður Ari & Fourier-ummyndun á lagi eftir Keikó  \\ \hline
       Teresa Ann & Stefnuháður núningur  \\ \hline
       Þorsteinn Wilhelm & Damian Lillard: "The Wave"  \\ \hline
    \end{tabular}
\end{table}

\begin{table}[H]
    \centering
    \begin{tabular}{|l|l|}
    \hline  \textbf{Nafn} & \textbf{Efni} \\ \hline \hline
       Alex Orri  & Ofurleiðarar  \\ \hline
       Benedikt Kristinn & Teygjustökk \\ \hline
       Einar Atli & Mæling Eratosþenesar á geisla jarðarinnar  \\ \hline
       Einar Andri & Geimflaugar  \\ \hline
       Einar Skúli & Regnboginn \\ \hline
       Guðmundur Tómas & Hvernig bjargar Júpíter okkur?  \\ \hline
       Haraldur & Vagga Newtons \\ \hline
       Hjörtur Viðar & Fallhlífarstökk   \\ \hline
       Hulda Eir & Fjarreikistjörnur  \\ \hline
       Ingibjörg Brynja & Skammtatölvur \\ \hline
       Jovan & Hið fullkomna körfuboltaskot  \\ \hline
       Ólafur Breki & Schrödinger-jafnan \\ \hline
       Sandra & Norðurljós \\ \hline
       Stefán Árni & Badmintonkúlur \\ \hline
       Sædís Ósk & Geimgeislar \\ \hline
       Ylja Karen & Maxwell djöfullinn og óreiða  \\ \hline
       Þóra Dís & Hvernig hlaupa eðlur á vatni?  \\ \hline
    \end{tabular}
\end{table}


\newpage


\section*{Mætingarlisti}

\begin{table}[H]
    \centering
    \begin{tabular}{|c|l|c|c|c|c|c|c|}
    \hline 
    \textbf{Nr.} & \textbf{Nafn} & \textbf{Mán} & \textbf{Þri} & \textbf{Mið} & \textbf{Fim} & \textbf{Fös} \\ \hline \hline
       1 & Aron Dimas  & & & & &  \\ \hline
       2 & Bjarki Þór & & & & &   \\ \hline
       3 & Davíð Freyr & & & & &  \\ \hline
       4 & Davíð Þór & & & & &  \\ \hline
       5 & Eðvald Þór & & & & &  \\ \hline
       6 & Egill Grétar & & & & &  \\ \hline
       7 & Eiður & & & & &  \\ \hline
       8 & Elísa Inger & & & & &  \\ \hline
       9 & Eva Mítra & & & & &  \\ \hline
       10 & Gunnar Snorri & & & & &  \\ \hline
       11 & Hannes Ísberg & & & & &  \\ \hline
       12 & Helgi Níels & & & & &  \\ \hline
       13 & Hildur & & & & &  \\ \hline
       14 & Ingi Hrafn & & & & &  \\ \hline
       15 & Jökull & & & & &  \\ \hline
       16 & Karitas Telma & & & & &  \\ \hline
       17 & Kristín Ingibjörg & & & & &  \\ \hline
       18 & Kristján Dagur & & & & &  \\ \hline
       19 & Magnús Nói & & & & &  \\ \hline
       20 & Mónika Andjani & & & & &  \\ \hline
       21 & Oddur & & & & &  \\ \hline
       22 & Ólafur Björn & & & & &  \\ \hline
       23 & Ragnar Björn & & & & &  \\ \hline
       24 & Sigurður Ari & & & & &  \\ \hline
       25 & Teresa Ann & & & & &  \\ \hline
       26 & Þorsteinn Wilhelm & & & & &  \\ \hline
    \end{tabular}
\end{table}

\begin{table}[H]
    \centering
    \begin{tabular}{|c|l|c|c|c|c|c|c|}
    \hline 
    \textbf{Nr.} & \textbf{Nafn} & \textbf{Mán} & \textbf{Þri} & \textbf{Mið} & \textbf{Fim} & \textbf{Fös} \\ \hline \hline
       1 & Alex Orri  & & & & &  \\ \hline
       2 & Benedikt Kristinn & & & & &   \\ \hline
       3 & Einar Atli & & & & &  \\ \hline
       4 & Einar Andri & & & & &  \\ \hline
       5 & Einar Skúli & & & & &  \\ \hline
       6 & Guðmundur Tómas & & & & &  \\ \hline
       7 & Haraldur & & & & &  \\ \hline
       8 & Hjörtur Viðar & & & & &  \\ \hline
       9 & Hulda Eir & & & & &  \\ \hline
       10 & Ingibjörg Brynja & & & & &  \\ \hline
       11 & Jovan & & & & &  \\ \hline
       12 & Ólafur Breki & & & & &  \\ \hline
       13 & Sandra & & & & &  \\ \hline
       14 & Stefán Árni & & & & &  \\ \hline
       15 & Sædís Ósk & & & & & \\ \hline
       16 & Ylja Karen & & & & &  \\ \hline
       17 & Þóra Dís & & & & &  \\ \hline
    \end{tabular}
\end{table}

\newpage

\subsection*{Y-bekkurinn}

\begin{table}[H]
    \centering
    \begin{tabular}{|c|l|c|c|c|c|c|c|c|c|c|c|c|c|}
    \hline 
    \textbf{Nr.} & \textbf{Nafn} & \phantom{Mán} & \phantom{Mán} & \phantom{Mán} & \phantom{Mán} & \phantom{Mán} & \phantom{Mán} & \phantom{Mán} & \phantom{Mán} & \phantom{Mán} & \phantom{Mán} & \phantom{Mán} & \phantom{Mán} \\ \hline \hline
       1 & Aron Dimas  & 3 & & & & & & & & & & &  \\ \hline
       2 & Bjarki Þór & 0 & 2 & & & & & & & & &  &  \\ \hline
       3 & Davíð Freyr & 0 & & & & & & & & & & &  \\ \hline
       4 & Davíð Þór & & & & & & & & & & &  & \\ \hline
       5 & Eðvald Þór & 1 & 3 & 2 & 0 & & & & & & & & \\ \hline
       6 & Egill Grétar & 2 & 3 & 2 & 2 & & & & & & & &  \\ \hline
       7 & Eiður & 0 & 2 & & & & & & & & & & \\ \hline
       8 & Elísa Inger & 3 & 2 & 3 & 3 & & & & & & & &  \\ \hline
       9 & Eva Mítra & 3 & 2 & 2 & & & & & & & & &  \\ \hline
       10 & Gunnar Snorri & 2 & 2 & & & & & & & & &  &  \\ \hline
       11 & Hannes Ísberg & 2 & & & & & & & & & & &  \\ \hline
       12 & Helgi Níels & 0 & 0 & & & & & & & & & &  \\ \hline
       13 & Hildur & 1 & 4 & 2 & 2 & 3 & 3 & & & & & &  \\ \hline
       14 & Ingi Hrafn & 2 & 1 & 0,5 & 2 & & & & & & & &  \\ \hline
       15 & Jökull & 0,5 & 2 & 3 & 3 & & & & & & & & \\ \hline
       16 & Karitas Telma & 2 & 1 & & & & & & & & &  &  \\ \hline
       17 & Kristín Ingibjörg & 2 & 2 & 0 & 2 & & & & & & &  &  \\ \hline
       18 & Kristján Dagur & 0,5 & 0,4 & 2 & 2 & & & & & & &  & \\ \hline
       19 & Magnús Nói & 1 & 0 & 2 & & & & & & & & & \\ \hline
       20 & Mónika Andjani & 0 & 1 & 2 & 2 & & & & & & & &  \\ \hline
       21 & Oddur & 2 & & & & & & & & & & &  \\ \hline
       22 & Ólafur Björn & 0 & 3 & & & & & & & & & &  \\ \hline
       23 & Ragnar Björn & 2 & 2 & 2 & & & & & & & & &  \\ \hline
       24 & Sigurður Ari & 2 & 1 & 2 & & & & & & & & & \\ \hline
       25 & Teresa Ann & 1 & & & & & & & & & & & \\ \hline
       26 & Þorsteinn Wilhelm & 1 & 2 & 2 & 2 & 2 & & & & & & &  \\ \hline
    \end{tabular}
\end{table}

\subsection*{X-bekkurinn}

\begin{table}[H]
    \centering
    \begin{tabular}{|c|l|c|c|c|c|c|c|c|c|c|c|c|c|}
    \hline 
    \textbf{Nr.} & \textbf{Nafn} & \phantom{Mán} & \phantom{Mán} & \phantom{Mán} & \phantom{Mán} & \phantom{Mán} & \phantom{Mán} & \phantom{Mán} & \phantom{Mán} & \phantom{Mán} & \phantom{Mán} & \phantom{Mán} & \phantom{Mán} \\ \hline \hline
       1 & Alex Orri  & 1 & 0 & 1 & 1 & & & & & & & &  \\ \hline
       2 & Benedikt Kristinn & 2 & 2 & 2 & 2 & & & & & & &  &  \\ \hline
       3 & Einar Atli & 3 & 2 & 2 & & & & & & & & &  \\ \hline
       4 & Einar Andri & 3 & 2 & 4 & 1 & 2 & 2 & & & & &  & \\ \hline
       5 & Einar Skúli & 0 & 1 & & & & & & & & & & \\ \hline
       6 & Guðmundur Tómas & 1 & 0 & 0 & 2 & 2 & & & & & & &  \\ \hline
       7 & Haraldur & 2 & 0 & 0 & 2 & 2 & & & & & & & \\ \hline
       8 & Hjörtur Viðar & 2 & 1 & 0 & 1 & & & & & & & &  \\ \hline
       9 & Hulda Eir & 1 & 2 & 1 & & & & & & & & &  \\ \hline
       10 & Ingibjörg Brynja & 2 & 2 & & & & & & & & &  &  \\ \hline
       11 & Jovan & 2 & 2 & 2 & 2 & & & & & & & &  \\ \hline
       12 & Ólafur Breki & 3 & 2 & 1 & 2 & 1 & & & & & & &  \\ \hline
       13 & Sandra & 2 & 0 & 2 & 0 & & & & & & & &  \\ \hline
       14 & Stefán Árni & 2 & 0 & 2 & 1 & 2 & & & & & & &  \\ \hline
       14 & Sædís Ósk & 2  &  2 &  &  & & & & & & & & \\ \hline
       16 & Ylja Karen & 1 & 2 & 0 & 2 & 1 & & & & & & & \\ \hline
       17 & Þóra Dís & 3 & 2 & & & & & & & & &  &  \\ \hline
    \end{tabular}
\end{table}

\newpage

\begin{tikzpicture}[remember picture, overlay]

\tikzset{normal lines/.style={gray, very thin}} 
\tikzset{margin lines/.style={gray, thick}} 
\tikzset{mm lines/.style={gray, ultra thin}} 
\tikzset{strong lines/.style={black, very thin}} 
\tikzset{master lines/.style={black, very thick}} 
\tikzset{dashed master lines/.style={loosely dashed, black, very thick}} 

\node at ([xshift=12.95mm, yshift=9.5mm] current page.south west){
  \begin{tikzpicture}[remember picture, overlay]

    \draw[style=mm lines,step=1mm] (0,0) grid +(19cm,26cm); 
    \draw[style=strong lines,step=1cm] (0,0) grid +(19cm,26cm); 

  \end{tikzpicture}
};
\end{tikzpicture}

\newpage

\section*{Tímaskýrsla í eðlisfræði}

\subsection*{Titill}

Hefðbundinn titill samkvæmt þeim venjum sem tíðkast í íslenskum stíl.

\subsection*{Inngangur}

Tilgreina markmið tilraunarinnar, hvenær tilraunin var framkvæmd og samstarfsmenn.

\subsection*{Uppstilling og framkvæmd}

Gefa greinargóða lýsingu á uppstillingu tilraunarinnar ásamt lýsingu á því hvernig tilraunin var framvkæmd (best að gera það með mynd!). Hér er æskilegt að kynna allar þær stærðir sem koma fyrir í tilrauninni.

\subsection*{Fræði}

Lýsa fræðinni sem að tilraunin byggir á. Setja fram fræðilegt líkan í samræmi við uppstillingu og skýra hvernig unnt er að ráða samband stærðanna sem koma fyrir í líkaninu út frá mælingum.

\subsection*{Úrvinnsla og niðurstöður} 

Mælingar settar skipulega fram í merktum töflum með óvissum og SI-einingum mældra stærða í viðeigandi dálkum. Fyrir neðan töflurnar ber að skýra innihald þeirra með stuttri lýsingu. Því næst kemur handteiknað graf. Handteiknuð gröf eiga að hafa vel merkta ása með tilheyrandi SI-einingum stærðanna sem koma fyrir á gröfunum. Óvissa mælipuntanna á að sjást greinilega á grafinu. Fyrir neðan gröfin ber að skýra innihald þeirra með stuttri lýsingu. Á handteiknaða grafinu er ætlast til þess að þið reiknið hallatölu bestu línunnar í gegnum gagnasettið ásamt því að þið metið óvissu hallatölunnar. Einungis er ætlast til þess að nemendur handteikni eitt graf úr tilrauninni (gefin verða bónusstig fyrir fleiri gröf). Ef að tilraunin hefur fleiri en eitt graf þá megið þið gefa niðurstöðuna af grafinu sem að Logger Pro gaf ykkur í hinum gröfunum (það graf sem þið handteiknið ekki). Loks koma útreikningar (með óvissu) á þeim stærðum sem við sækjumst eftir í samræmi við fræðina.

\subsection*{Ályktanir}

Ofangreindar niðurstöður eru dregnar saman skipulega og bornar saman við upphaflega líkanið sem sett var fram. Óháð því hvort að niðurstöðurnar passi innan óvissumarka þarf að skýra hlestu óvissuþætti.

\subsection*{Síðast en ekki síst}

Skýrslan er undirrituð neðst með nafni, staðsetningu og dagsetningu. Merkið handteiknuð gröf með nafni.


\newpage

\section*{Skýrslur}

\begin{tcolorbox}

Heildarstigafjöldinn fyrir skýrsluna er \SI{100}{stig}.

\subsection*{Inngangur (5 stig)}

Fyrir inngang eru gefin eftirfarandi stig:

\begin{table}[H]
    \centering
    \begin{tabular}{|c|l|}
    \hline
       \textbf{Stig}  & \textbf{Hugmynd}   \\ \hline \hline
        1  & Dagsetning. \\ \hline
        1  & Samstarfsfélagar. \\ \hline
        3  & Yfirlýst markmið í samræmi við tilraunina. \\ \hline
    \end{tabular}
\end{table}

Ef að nemandi endurtekur markmiðslýsinguna í verkseðlinum þá eru það $-2$ stig í frádrátt.

\subsection*{Uppstilling og framkvæmd (20 stig)}

\begin{table}[H]
    \centering
    \begin{tabular}{|c|l|}
    \hline
       \textbf{Stig}  & \textbf{Hugmynd}   \\ \hline \hline
        5  & Skýringarmynd sem sýnir uppstillinguna. \\ \hline
        5  & Allar breytistærðir hafa fengið algebrulegt tákn og skýringu. \\ \hline
        3  & Lýsing á því hvað var verið að skoða í tilrauninni. \\ \hline
        4  & Lýsing á því hvað var mælt í tilrauninni og hvernig það var mælt. \\ \hline
        3  & Tækjabúnaðurinn hefur verið skýrður í texta eða út frá mynd. \\ \hline
    \end{tabular}
\end{table}

Hér eru veitt bónusstig fyrir eftirfarandi:

\begin{table}[H]
    \centering
    \begin{tabular}{|c|l|}
    \hline
       \textbf{Stig}  & \textbf{Hugmynd}   \\ \hline \hline
       $+3$ & Áberandi góðar og lýsandi skýringarmyndir.  \\ \hline
        $+2$ & Lýsing á mælitækni sem er frábrugðin þeirri í verkseðlinum.  \\ \hline
        $+1$  & Tilvísanir í skýringarmyndina á viðeigandi stað í textanum. \\ \hline
    \end{tabular}
\end{table}

Eftirfarandi er frádráttarvert:

\begin{table}[H]
    \centering
    \begin{tabular}{|c|l|}
    \hline
       \textbf{Stig}  & \textbf{Hugmynd}   \\ \hline \hline
        -3  & Notar mynd úr verkseðlinum sem skýringarmynd. \\ \hline
        -3  & Endurtekning á því sem stendur í verkseðlinum. \\
        \hline
        -1  & Fyrir hverja breytistærð sem vantar að skýra með algebrulegu tákni. \\ \hline
    \end{tabular}
\end{table}



\subsection*{Fræði (25 stig)}

Í fræðihlutanum eru veitt stig fyrir eftirfarandi:

\begin{table}[H]
    \centering
    \begin{tabular}{|c|l|}
    \hline
       \textbf{Stig}  & \textbf{Hugmynd}   \\ \hline \hline
        2  & Jöfnur eru númeraðar svo hægt sé að vísa í þær.  \\ \hline
        5  & Líkanið sem verið er að sannreyna kemur fyrir. \\\hline
        8  & Útleiðsla á líkaninu út frá grunnlögmálum. \\\hline
        5  & Nemandi bætir við í fræðihlutann. \\\hline
        5  & Samanburður við aðhvarfslíkanið og stuðla þess. \\\hline
    \end{tabular}
\end{table}

\end{tcolorbox}

\newpage

\begin{tcolorbox}

Í fræðihlutanum eru frádrættir fyrir eftirfarandi:

\begin{table}[H]
    \centering
    \begin{tabular}{|c|l|}
    \hline
       \textbf{Stig}  & \textbf{Hugmynd}   \\ \hline \hline
        -5  & Endurtekning á því sem stendur í verkseðlinum án viðbóta.  \\ \hline
        -2  & Líkanið sem verið er að sannreyna kemur ekki fyrir.
        \\\hline
        -3  & Áttar sig ekki á tengsl stuðlanna í líkaninu við grafið.
        \\\hline
        -3  & Notar breytistærðir sem hafa ekki verið skilgreindar áður í skýrslunni.
        \\\hline
        -2  & Missamræmi í breytuheitum.
        \\\hline
    \end{tabular}
\end{table}

\subsection*{Töflur og mælingar (10 stig)}

Fyrir töflur og mælingar eru gefin eftirfarandi stig:

\begin{table}[H]
    \centering
    \begin{tabular}{|c|l|}
    \hline
       \textbf{Stig}  & \textbf{Hugmynd}   \\ \hline \hline
        3  & Efsta línan inniheldur algebrulegar mælistærðir og einingar þeirra. \\ \hline
        3  & Mælistærðirnar koma fyrir ásamt óvissu þeirra í töflunni. \\ \hline
        2  & Fleiri en 7 gildi eru í töflunni. \\ \hline
        2  & Undirtitill sem útskýrir hvað taflan sýnir. \\ \hline
    \end{tabular}
\end{table}

Nemendur fá frádrátt upp á $-7$ stig ef að þeir sýna aðeins ljósmynd eða skjáskot af töflu. Það eru veitt allt að $+5$ bónusstig ef að nemandi reiknar aðrar breytistærðir heldur en þær sem koma fram í verkseðlinum ásamt óvissu þeirra (það er aðeins hægt að fá þessi stig ef maður er að bæta við líkanið).


\subsection*{Gröf (10 stig)}

Fyrir gröf eru gefin eftirfarandi stig:

\begin{table}[H]
    \centering
    \begin{tabular}{|c|l|}
    \hline
       \textbf{Stig}  & \textbf{Hugmynd}   \\ \hline \hline
        1  & Lögun grafsins hefur verið skýrð áður en grafið var gert. \\ \hline
         1  & Undirtitill sem útskýrir hvað grafið er af. \\ \hline
        2  & Algebrulegt tákn með einingum á ásum. \\ \hline
        2  & Óvissur mælipunktanna á grafinu sjást. \\ \hline
        2  & Grafið sýnir 7 eða fleiri mælipunkta. \\ \hline
        2  & Aðhvarfsstuðlar líkansins sjást skýrt á grafinu eða undir því. \\ \hline
    \end{tabular}
\end{table}


Hér eru veitt bónusstig fyrir eftirfarandi:

\begin{table}[H]
    \centering
    \begin{tabular}{|c|l|}
    \hline
       \textbf{Stig}  & \textbf{Hugmynd}   \\ \hline \hline
         +2  & Nemandi sýnir ólínulegt graf og skýrir tengsl við fræðina. \\ \hline
        +2  & Nemandi hefur útbúið grafið í Python, R eða Gnuplot. \\ \hline
        +1  & Nemandinn hefur sett myndina inn á svg formi.  \\ \hline
    \end{tabular}
\end{table}

Það eru veitt allt að $+5$ bónusstig fyrir að gera grafið í Python, R eða Gnuplot. Hinsvegar er $-3$ stiga frádráttur fyrir að gera grafið í Excel, ásamt eftirfarandi:

Það eru síðan frádrættir fyrir eftirfraandi:

\begin{table}[H]
    \centering
    \begin{tabular}{|c|l|}
    \hline
       \textbf{Stig}  & \textbf{Hugmynd}   \\ \hline \hline
         -7  & Ljósmynd af grafi. \\ \hline
        -3  & Gagnatafla vinstra eða hægra meginn við grafið. \\ \hline
        -2  & Vantar óvissur á aðhvarfsstuðla grafsins. \\ \hline
        -3 & Grafið er gert í Excel. \\ \hline
    \end{tabular}
\end{table}

\end{tcolorbox}

\newpage

\begin{tcolorbox}

\subsection*{Óvissureikningar (15 stig)}

Fyrir óvissureikninga eru gefin eftirfarandi stig:

\begin{table}[H]
    \centering
    \begin{tabular}{|c|l|}
    \hline
       \textbf{Stig}  & \textbf{Hugmynd}   \\ \hline \hline
        3  & Allar breytistærðir sem koma fyrir hafa mat á óvissu. \\ \hline
        2  & Útkýring á mati á óvissu. \\ \hline
        2  & Rétt meðhöndlun á marktækum stöfum hjá mælistærðum með óvissu. \\ \hline
        4  & Sýnir algebrulega hvernig óvissa afleiddra stærða er reiknuð. \\ \hline
        2  & Töluleg gildi á óvissu afleiddra stærða.  \\ \hline
        2 & Óvissureikningur af gerðinni $\Delta f(x) = f'(x) \Delta x$.  \\ \hline 
    \end{tabular}
\end{table}


\subsection*{Úrvinnsla (5 stig)}

\begin{table}[H]
    \centering
    \begin{tabular}{|c|l|}
    \hline
       \textbf{Stig}  & \textbf{Hugmynd}   \\ \hline \hline
       $+5$ & Ákvarðar viðeigandi stærð út frá aðhvarfsstuðlum grafsins með óvissu.   \\ \hline
    \end{tabular}
\end{table}





\subsection*{Ályktanir (10 stig)}

Fyrir ályktunarlið eru gefin eftirfarandi stig:

\begin{table}[H]
    \centering
    \begin{tabular}{|c|l|}
    \hline
       \textbf{Stig}  & \textbf{Hugmynd}   \\ \hline \hline
        1  &  Umorðar og tengir niðurstöðuna við markmiðið. \\ \hline
        1  & Tekur saman mæliniðurstöður tilraunarinnar á skipulegan hátt. \\ \hline
        2  & Athugar hvort að niðurstaðan sé innan óvissumarka við viðtekið gildi. \\ \hline
        2  & Nefnir helstu óvissuþætti í tilrauninni og hvernig mætti bæta það. \\ \hline
        4 & Setur niðurstöðuna í samhengi og reynir að álykta hvaða merkingu hún hefur. \\ \hline
    \end{tabular}
\end{table}

Frádráttur um $-1$ stig er fyrir lýsingar á borð við "Tilraunin heppnaðist vel" og "Markmiðinu var náð". Hægt er að fá bónusstig fyrir eftirfarandi:

\begin{table}[H]
    \centering
    \begin{tabular}{|c|l|}
    \hline
       \textbf{Stig}  & \textbf{Hugmynd}   \\ \hline \hline
        +5  &  Bætir við framhaldi af tilrauninni. \\ \hline
    \end{tabular}
\end{table}

\section*{Annað}

Almennir frádrættir eru fyrir eftirfarandi hluti:

\begin{table}[H]
    \centering
    \begin{tabular}{|c|l|}
    \hline
       \textbf{Stig}  & \textbf{Hugmynd}   \\ \hline \hline
        -5  & Hefur stutt sig of mikið við verkseðilinn við gerð skýrslunnar. \\ \hline
        -100  & Nemandi skilar eins skýrslu og annar nemandi. \\ \hline
    \end{tabular}
\end{table}
\end{tcolorbox}

\newpage


\begin{tcolorbox}

\section*{Stigagjafarskema fyrir tímaskýrslu (Drög)}

Almennt eru gefnir frádrættir fyrir endurtekningu á efninu sem að kemur fram í verkseðlinum. Það efni sem að kemur beint upp úr verkseðlinum er ekki stigabært í samræmi við stigagjafarskemað.

\subsection*{Inngangur (5 stig)}

\begin{table}[H]
    \centering
    \begin{tabular}{|c|l|}
    \hline
       \textbf{Stig}  & \textbf{Hugmynd}   \\ \hline \hline
        1  & Dagsetning. \\ \hline
        1  & Samstarfsfélagar. \\ \hline
        3  & Yfirlýst markmið í samræmi við tilraunina. \\ \hline
    \end{tabular}
\end{table}

\subsection*{Uppstilling og framkvæmd (12 stig)}

\begin{table}[H]
    \centering
    \begin{tabular}{|c|l|}
    \hline
       \textbf{Stig}  & \textbf{Hugmynd}   \\ \hline \hline
        5  & Skýringarmynd sem sýnir uppstillinguna. \\ \hline
        4  & Allar breytistærðir hafa fengið algebrulegt tákn og skýringu. \\ \hline
        3  & Stutt lýsing á því hvað var mælt í tilrauninni og hvernig það var mælt. \\ \hline
    \end{tabular}
\end{table}

\subsection*{Fræði (12 stig)}

\begin{table}[H]
    \centering
    \begin{tabular}{|c|l|}
    \hline
       \textbf{Stig}  & \textbf{Hugmynd}   \\ \hline \hline
        1  & Jöfnur eru númeraðar svo hægt sé að vísa í þær.  \\ \hline
        5 & Líkanið sem verið er að sannreyna kemur fyrir. \\\hline
        6  & Samanburður við aðhvarfslíkanið og stuðla þess. \\\hline
    \end{tabular}
\end{table}

Í fræðihlutanum eru bónusstig fyrir eftirfarandi:

\begin{table}[H]
    \centering
    \begin{tabular}{|c|l|}
    \hline
       \textbf{Stig}  & \textbf{Hugmynd}   \\ \hline \hline
        +4  & Útleiðsla á líkaninu (sem kemur ekki fram í verkseðlinum).
        \\\hline
    \end{tabular}
\end{table}

\subsection*{Töflur og mælingar (7 stig)}

\begin{table}[H]
    \centering
    \begin{tabular}{|c|l|}
    \hline
       \textbf{Stig}  & \textbf{Hugmynd}   \\ \hline \hline
        1  & Efsta línan inniheldur algebrulegar mælistærðir og einingar þeirra. \\ \hline
        2 & Mælistærðirnar koma fyrir ásamt óvissu þeirra í töflunni. \\ \hline
        2  & Fleiri en 7 gildi eru í töflunni. \\ \hline
        2  & Undirtitill sem útskýrir hvað taflan sýnir. \\ \hline
    \end{tabular}
\end{table}

\end{tcolorbox}

\newpage

\begin{tcolorbox}

\subsection*{Handteiknað graf (30 stig)}

Fyrir handteiknaða grafið eru gefin eftirfarandi stig:

\begin{table}[H]
    \centering
    \begin{tabular}{|c|l|}
    \hline
       \textbf{Stig}  & \textbf{Hugmynd}   \\ \hline \hline
        2  & Lögun grafsins hefur verið skýrð áður en grafið var gert. \\ \hline
         1  & Undirtitill sem útskýrir hvað grafið er af. \\ \hline
        2  & Algebrulegt tákn með einingum á ásum. \\ \hline
        2  & Kvarðinn hefur verið valinn skynsamlega þ.a.~67\% af grafinu nýtist. \\ \hline
        3  & Óvissur mælipunktanna á grafinu sjást. \\ \hline
        4  & Grafið sýnir 7 eða fleiri mælipunkta. \\ \hline
        1  & Punktum er sleppt ef þeir falla ekki að línunni. \\ \hline
        2  & Það er skýrt út frá hvaða tveimur punktum hallatalan er reiknuð. \\ \hline
        6  & Hallatalan er reiknuð. \\ \hline
        4  & Óvissan á hallatölunni er reiknuð. \\ \hline
        2  & Skurðpunktur grafsins er reiknaður eða metinn út frá grafinu. \\ \hline
        1  & Óvissan á skurðpunktinum er reiknuð eða metin út frá grafinu.  \\ \hline
    \end{tabular}
\end{table}

Fyrir handteiknuð gröf eru veitt bónusstig fyrir eftirfarandi:


\begin{table}[H]
    \centering
    \begin{tabular}{|c|l|}
    \hline
       \textbf{Stig}  & \textbf{Hugmynd}   \\ \hline \hline
        +1  & Hallatalan er reiknuð út frá punktum sem eru ekki hluti af gagnasettinu.
        \\\hline
        +3  & Tvö gröf eru handteiknuð og aðhvarfsstuðlar þeirra reiknaðir.
        \\\hline
        +4   & Þrjú gröf eru handteiknuð og aðhvarfsstuðlar þeirra reiknaðir.
        \\\hline
    \end{tabular}
\end{table}

Fyrir handteiknuð gröf eru frádrættir fyrir eftirfarandi:


\begin{table}[H]
    \centering
    \begin{tabular}{|c|l|}
    \hline
       \textbf{Stig}  & \textbf{Hugmynd}   \\ \hline \hline
        -2  & Punktarnir eru ekki í samræmi við gagnasettið.
        \\\hline
        -2  & Óvissumörkin á punktunum eru ekki í samræmi við staðhæfða gagnaóvissu.
        \\\hline
        -2   & Hallatalan hefur verið reiknuð út frá slæmum punktum.
        \\\hline
    \end{tabular}
\end{table}

\subsection*{Óvissur og óvissureikningar (10 stig)}

Fyrir óvissureikninga eru gefin eftirfarandi stig:

\begin{table}[H]
    \centering
    \begin{tabular}{|c|l|}
    \hline
       \textbf{Stig}  & \textbf{Hugmynd}   \\ \hline \hline
        2  & Allar breytistærðir sem koma fyrir hafa mat á óvissu. \\ \hline
        2  & Rétt meðhöndlun á marktækum stöfum hjá mælistærðum með óvissu. \\ \hline
        6  & Sýnir algebrulega hvernig óvissa afleiddra stærða er reiknuð. \\ \hline
    \end{tabular}
\end{table}


\end{tcolorbox}

\newpage

\begin{tcolorbox}


\subsection*{Úrvinnsla (12 stig)}

\begin{table}[H]
    \centering
    \begin{tabular}{|c|l|}
    \hline
       \textbf{Stig}  & \textbf{Hugmynd}   \\ \hline \hline
       $2$ & Ákvarðar algebrulega viðeigandi stærðir út frá aðhvarfsstuðlum grafsins.   \\
       \hline
       $5$ & Reiknar tölulegt gildi á stærðunum sem á að ákvarða.  \\ \hline
       $5$ & Metur óvissuna á niðurstöðunni.  \\ \hline
    \end{tabular}
\end{table}





\subsection*{Ályktanir (12 stig)}


\begin{table}[H]
    \centering
    \begin{tabular}{|c|l|}
    \hline
       \textbf{Stig}  & \textbf{Hugmynd}   \\ \hline \hline
        2  &  Umorðar og tengir niðurstöðuna við markmiðið. \\ \hline
        1  & Tekur saman mæliniðurstöður tilraunarinnar á skipulegan hátt. \\ \hline
        2  & Athugar hvort að niðurstaðan sé innan óvissumarka við viðtekið gildi. \\ \hline
        2  & Nefnir helstu óvissuþætti í tilrauninni og hvernig mætti bæta það. \\ \hline
        5 & Setur niðurstöðuna í samhengi og reynir að álykta hvaða merkingu hún hefur. \\ \hline
    \end{tabular}
\end{table}

Frádráttur um $-1$ stig er fyrir lýsingar á borð við \emph{,,Tilraunin heppnaðist vel''} eða \emph{,,Markmiðinu var náð''}. Hægt er að fá bónusstig fyrir eftirfarandi:

\begin{table}[H]
    \centering
    \begin{tabular}{|c|l|}
    \hline
       \textbf{Stig}  & \textbf{Hugmynd}   \\ \hline \hline
        +5  &  Bætir við sjálfstæðu framhaldi af tilrauninni (sem kemur ekki fram í verkseðlinum). \\ \hline
    \end{tabular}
\end{table}

\end{tcolorbox}

\newpage

\begin{tcolorbox}

\section*{Stigagjafarskema fyrir próf í verklegu (Drög)}

\subsection*{Liður 1.1. (10 stig)}

Hver liður liðsins gaf um það bil 1-2 stig nema síðasti liðurinn sem gaf 3 stig.

\subsection*{Liður 1.2. (10 stig)}

Réttu hornin sem að $Y$-bekkurinn átti að mæla voru:
\begin{align*}
    \theta_1 \in \left[ 33;39 \right], \hspace{0.5cm} \theta_2 \in [21;27].
\end{align*}
og hornin sem að $X$-bekkurinn átti að mæla voru:
\begin{align*}
    \theta_1 \in \left[ 40;46 \right], \hspace{0.5cm} \theta_2 \in [24;30] \hspace{0.5cm} \text{eða} \hspace{0.5cm} \theta_3 \in \left[ 30;36 \right], \hspace{0.5cm} \theta_4 \in [48;54]
\end{align*}
Hinsvegar virtust flestir ekki hafa skilið hvað ,,þverill á yfirborð hlutarins`` merkir og það var því í algjörum undartekningum sem að nemendur mældu rétt horn! Stigagjöfin var síðan eftirfarandi:

\begin{table}[H]
    \centering
    \begin{tabular}{|c|l|}
    \hline
       \textbf{Stig}  & \textbf{Hugmynd}   \\ \hline \hline
        1  & Fyrir hvert horn sem var mælt sem var innan leyfilegu bilanna. \\ \hline
        1  & Ef bæði hornin voru rétt.  \\ \hline
        1  & Staðhæfir að $n_{\text{loft}} = 1$. \\ \hline
        2  & Einangra fyrir $n_2 = \frac{\sin\theta_1}{\sin\theta_2}$. \\ \hline
        1  & Tölulegt gildi $n_2$. \\ \hline
        2  & Óvissan reiknuð samkvæmt óvissureglunni. \\ \hline
        1  & Man að breyta óvissunni á horninu í rad þegar óvissan er reiknuð. \\ \hline
    \end{tabular}
\end{table}

\subsection*{Liður 2.1. (5 stig)}

\begin{itemize}
    \item Umritar jöfnuna á leyfilegt form.
\end{itemize}

\subsection*{Liður 2.2. (15 stig)}

\begin{table}[H]
    \centering
    \begin{tabular}{|c|l|}
    \hline
       \textbf{Stig}  & \textbf{Hugmynd}   \\ \hline \hline
        8  & Bætir við dálkum í samræmi við liðinn á undan. \\ \hline
        5  & Bætir við tilheyrandi óvissudálki.  \\ \hline
        2  & Óvissan passar alls staðar við mælistærðina. \\ \hline
    \end{tabular}
\end{table}


\subsection*{Liður 2.3. (15 stig)}

\begin{table}[H]
    \centering
    \begin{tabular}{|c|l|}
    \hline
       \textbf{Stig}  & \textbf{Hugmynd}   \\ \hline \hline
         1  & Undirtitill sem útskýrir hvað grafið er af. \\ \hline
        2  & Algebrulegt tákn með einingum á ásum. \\ \hline
        2  & Kvarðinn hefur verið valinn skynsamlega þ.a.~67\% af grafinu nýtist. \\ \hline
        4  & Grafið sýnir 6 eða fleiri mælipunkta. \\ \hline
        3  & Óvissa á $x$-ás í samræmi við töfluna í liðnum á undan. \\ \hline
        3  & Óvissa á $y$-ás í samræmi við töfluna í liðnum á undan. \\ \hline
    \end{tabular}
\end{table}

\end{tcolorbox}

\begin{tcolorbox}

\subsection*{Liður 2.4. (20 stig)}

\begin{table}[H]
    \centering
    \begin{tabular}{|c|l|}
    \hline
       \textbf{Stig}  & \textbf{Hugmynd}   \\ \hline \hline
        2  & Það er skýrt út frá hvaða tveimur punktum hallatalan er reiknuð. \\ \hline
        4  & Hallatalan er reiknuð. \\ \hline
        4  & Óvissan á hallatölunni er reiknuð. \\ \hline
        2  & Hallatalan er borin saman við fyrsta liðin. \\ \hline
        2  & Hallatalan er umrituð til þess að ákvarða umbeðna stærð.  \\ \hline
        2  & Óvissan á umbeðnu stærðinni er metin algbrulega.  \\ \hline
        2  & Óvissan á umbeðnu stærðinni er metin tölulega.  \\ \hline
        2  & Samanburður við þekkt gildi.  \\ \hline
    \end{tabular}
\end{table}

\subsection*{Liður 3.1. (2 stig)}

\textit{Ef að nemendur byrjuðu á þessum verkefni og tókst ekki að klára verkefni 2 þá gildir stigagjöfin fyrir verkefni 2 í staðinn fyrir þá nemendur þar sem að sú stigagjöf er gjafmildari.}

\begin{itemize}
    \item Umritar jöfnuna á leyfilegt form.
\end{itemize}

\subsection*{Liður 3.2. (6 stig)}

\begin{table}[H]
    \centering
    \begin{tabular}{|c|l|}
    \hline
       \textbf{Stig}  & \textbf{Hugmynd}   \\ \hline \hline
        3  & Bætir við dálkum í samræmi við liðinn á undan. \\ \hline
        2  & Bætir við tilheyrandi óvissudálki.  \\ \hline
        1  & Óvissan passar alls staðar við mælistærðina. \\ \hline
    \end{tabular}
\end{table}

\subsection*{Liður 3.3. (17 stig)}

\begin{table}[H]
    \centering
    \begin{tabular}{|c|l|}
    \hline
       \textbf{Stig}  & \textbf{Hugmynd}   \\ \hline \hline
         1  & Undirtitill sem útskýrir hvað grafið er af. \\ \hline
        1  & Algebrulegt tákn með einingum á ásum. \\ \hline
        1  & Kvarðinn hefur verið valinn skynsamlega þ.a.~67\% af grafinu nýtist. \\ \hline
        1  & Grafið sýnir 6 eða fleiri mælipunkta. \\ \hline
        2  & Óvissan á ásum í samræmi við töfluna í liðnum á undan. \\ \hline
        1  & Það er skýrt út frá hvaða tveimur punktum hallatalan er reiknuð. \\ \hline
        2  & Hallatalan er reiknuð. \\ \hline
        2 & Óvissan á hallatölunni er reiknuð. \\ \hline
        2  & Hallatalan er umrituð til þess að ákvarða umbeðna stærð.  \\ \hline
        2  & Óvissan á umbeðnu stærðinni er metin algebrulega.  \\ \hline
        2  & Óvissan á umbeðnu stærðinni er metin tölulega.  \\ \hline
    \end{tabular}
\end{table}

\subsection*{Liður 3.4. (5 stig)}

\begin{table}[H]
    \centering
    \begin{tabular}{|c|l|}
    \hline
       \textbf{Stig}  & \textbf{Hugmynd}   \\ \hline \hline
         2  & Les rétt af grafinu sem er gefið. \\ \hline
         3  & Ber gildið saman við gildið sitt. \\ \hline
    \end{tabular}
\end{table}


\end{tcolorbox}