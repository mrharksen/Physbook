\chapter*{Hvernig á að leysa eðlisfræðidæmi}

Hér setjum við fram lista af hlutum sem er gott að hafa í huga þegar þið eruð að fást við eðlisfræðidæmi.


\begin{enumerate}[label = \textbf{Skref \arabic*.}]
    \item Lestu dæmið vandlega.
    
    \item Teiknaðu skýra mynd (helst eins stóra og þú hefur pláss fyrir).
    
    \item Áttaðu þig á stærðunum sem þú þekkir og stærðunum sem þú ert að reyna að finna og merktu á mynd.
    
    \item Veldu hnitakerfi eða viðmiðunarkerfi.
    
    \item Ákvarðaðu upphafsástand og lokaástand kerfisins sem þú ert að skoða.
    
    \item Koma kraftar fyrir í verkefninu? Þá er alltaf góð hugmynd að teikna kraftamynd.
    
    \item Er einhver varðveitt stærð? T.d. skriðþungi, orka eða hverfiþungi?
    
    \item Gefðu öllum tölunum sem koma fyrir í verkefninu stærðfræðilegt tákn. (t.d. $g = \SI{9.82}{m/s^2}$).
    
    \item Leystu verkefnið táknrænt.
\end{enumerate}

\section*{Að leysa táknrænt}

Ef við erum að leysa verkefni þá er betra að leysa verkefnin stærðfræðilega með táknunum sem fyrir koma í verkefninu. Það er góð þumalputtaregla að reyna að bíða með það eins lengi og hægt er að setja inn tölurnar (og helst bara í svarinu).

\begin{itemize}
    \item \textsc{Það er fljótlegra.} að margfalda táknin $g$ og $\ell$ saman heldur en að margfalda saman stærðirnar sem að stæðirnar standa fyrir og slá það inn á reiknivélina. Stundum þurfum við að framkvæma fimm eða tíu slíkar margfaldanir til þess að leysa dæmið og það er fljótt að stela af manni tíma ef maður þarf alltaf að slá inn á reiknivélina í hvert skipti.
    
    \item \textsc{Það er ólíklegra að maður geri klaufavillur.} ef maður reiknar með táknum frekar heldur en með tölum. Þannig tekur maður út allar klaufavillur eins og að slá inn $8$ á reiknivél þegar maður ætlaði að slá inn $9$.
    
    \item \textsc{Það er almennara.} því ef tölunum er síðan breytt, t.d. $\ell = \SI{2.4}{m}$ í staðinn fyrir $\ell = \SI{1.3}{m}$ þá er táknræna svarið óbreytt.
    
    \item \textsc{Við getum skoðað víddirnar.} Eftir að við höfum leyst dæmið er miklu auðveldara fyrir okkur að átta okkur á því hvort að við höfum gert einhverjar klaufavillur á leiðinni með því að skoða víddirnar á svarinu okkar.
    
    
    \item \textsc{Við getum skoðað jaðartilfelli.} Það er miklu auðveldara að skoða hvað gerist t.d. ef við breytum horninu $\theta$ frá því að vera $\ang{22}$ í að vera $\ang{90}$.
\end{itemize}

\newpage

\tableofcontents

\newpage